% Options for packages loaded elsewhere
\PassOptionsToPackage{unicode}{hyperref}
\PassOptionsToPackage{hyphens}{url}
%
\documentclass[
]{article}
\usepackage{amsmath,amssymb}
\usepackage{iftex}
\ifPDFTeX
  \usepackage[T1]{fontenc}
  \usepackage[utf8]{inputenc}
  \usepackage{textcomp} % provide euro and other symbols
\else % if luatex or xetex
  \usepackage{unicode-math} % this also loads fontspec
  \defaultfontfeatures{Scale=MatchLowercase}
  \defaultfontfeatures[\rmfamily]{Ligatures=TeX,Scale=1}
\fi
\usepackage{lmodern}
\ifPDFTeX\else
  % xetex/luatex font selection
\fi
% Use upquote if available, for straight quotes in verbatim environments
\IfFileExists{upquote.sty}{\usepackage{upquote}}{}
\IfFileExists{microtype.sty}{% use microtype if available
  \usepackage[]{microtype}
  \UseMicrotypeSet[protrusion]{basicmath} % disable protrusion for tt fonts
}{}
\makeatletter
\@ifundefined{KOMAClassName}{% if non-KOMA class
  \IfFileExists{parskip.sty}{%
    \usepackage{parskip}
  }{% else
    \setlength{\parindent}{0pt}
    \setlength{\parskip}{6pt plus 2pt minus 1pt}}
}{% if KOMA class
  \KOMAoptions{parskip=half}}
\makeatother
\usepackage{xcolor}
\usepackage[margin=1in]{geometry}
\usepackage{graphicx}
\makeatletter
\def\maxwidth{\ifdim\Gin@nat@width>\linewidth\linewidth\else\Gin@nat@width\fi}
\def\maxheight{\ifdim\Gin@nat@height>\textheight\textheight\else\Gin@nat@height\fi}
\makeatother
% Scale images if necessary, so that they will not overflow the page
% margins by default, and it is still possible to overwrite the defaults
% using explicit options in \includegraphics[width, height, ...]{}
\setkeys{Gin}{width=\maxwidth,height=\maxheight,keepaspectratio}
% Set default figure placement to htbp
\makeatletter
\def\fps@figure{htbp}
\makeatother
\setlength{\emergencystretch}{3em} % prevent overfull lines
\providecommand{\tightlist}{%
  \setlength{\itemsep}{0pt}\setlength{\parskip}{0pt}}
\setcounter{secnumdepth}{-\maxdimen} % remove section numbering
\ifLuaTeX
  \usepackage{selnolig}  % disable illegal ligatures
\fi
\usepackage{bookmark}
\IfFileExists{xurl.sty}{\usepackage{xurl}}{} % add URL line breaks if available
\urlstyle{same}
\hypersetup{
  pdftitle={A3: Incarceration},
  hidelinks,
  pdfcreator={LaTeX via pandoc}}

\title{A3: Incarceration}
\author{}
\date{\vspace{-2.5em}}

\begin{document}
\maketitle

\subsubsection{Introduction}\label{introduction}

All the data sets in this assignment are about incarceration either in
the state or in the country. I am going to find the average amount of
people incarcerated in the state and the average of people incarcerated
in Washington.The average could tell us how much the average a prison
can take. Also to see how much of the states has people incarcerated.

the data I ma going to find: What is the average value of my variable
across all the counties? what is the average in Washington? how many
rows does the data set have? how many columns does the data set have?

\subsubsection{Summary Information}\label{summary-information}

Write a summary paragraph of findings that includes the 5 values
calculated from your summary information R script

These will likely be calculated using your DPLYR skills, answering
questions such as:~

\begin{itemize}
\tightlist
\item
  What is the average value of my variable across all the counties (in
  the current year)?
\item
  Where is my variable the highest / lowest?
\item
  How much has my variable change over the last N years?
\end{itemize}

Feel free to calculate and report values that you find relevant. Again,
remember that the purpose is to think about how these measure of
incarceration vary by race.

\subsubsection{The Dataset}\label{the-dataset}

Who collected the data?\\
How was the data collected or generated?\\
Why was the data collected?\\
How many observations (rows) are in your data?\\
How many features (columns) are in the data?\\
What, if any, ethical questions or questions of power do you need to
consider when working with this data?\\
What are possible limitations or problems with this data? (at least 200
words)

\subsubsection{Trends Over Time Chart}\label{trends-over-time-chart}

Include a chart. Make sure to describe why you included the chart, and
what patterns emerged

The first chart that you will create and include will show the trend
over time of your variable/topic. Think carefully about what you want to
communicate to your user (you may have to find relevant trends in the
dataset first!). Here are some requirements to help guide your design:

\begin{itemize}
\tightlist
\item
  Show more than one, but fewer than \textasciitilde10 trends

  \begin{itemize}
  \tightlist
  \item
    This may mean showing the same measure for different locations or
    different racial groups. Think carefully about a meaningful
    comparison of locations (e.g., the top 10 counties in a state, top
    10 states, etc.)
  \end{itemize}
\item
  You must have clear x and y axis labels
\item
  The chart needs a clear title~
\item
  You need a legend for your different line colors and a clear legend
  title
\end{itemize}

When we say ``clear'' or ``human readable'' titles and labels, that
means that you should not just display the variable name.

Here's an example of how to run an R script inside an RMarkdown file:

\includegraphics{index_files/figure-latex/unnamed-chunk-1-1.pdf}

\subsubsection{Variable Comparison
Chart}\label{variable-comparison-chart}

Include a chart. Make sure to describe why you included the chart, and
what patterns emerged

The second chart that you will create and include will show how two
different (continuous) variables are related to one another. Again,
think carefully about what such a comparison means and what you want to
communicate to your user (you may have to find relevant trends in the
dataset first!). Here are some requirements to help guide your design:

\begin{itemize}
\tightlist
\item
  You must have clear x and y axis labels
\item
  The chart needs a clear title~
\item
  If you choose to add a color encoding (not required), you need a
  legend for your different color and a clear legend title
\end{itemize}

\subsubsection{Map}\label{map}

Include a chart. Make sure to describe why you included the chart, and
what patterns emerged

The last chart that you will create and include will show how a variable
is distributed geographically. Again, think carefully about what such a
comparison means and what you want to communicate to your user (you may
have to find relevant trends in the dataset first!). Here are some
requirements to help guide your design:

\begin{itemize}
\tightlist
\item
  Your map needs a title
\item
  Your color scale needs a legend with a clear label
\item
  Use a map based coordinate system to set the aspect ratio of your map
\end{itemize}

\end{document}
